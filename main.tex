\documentclass{report}

\usepackage[utf8]{inputenc}
\usepackage{eumat}
\usepackage[Conny]{fncychap}
\usepackage[bahasa]{babel}

% Rename Contents
\addto\captionsenglish{\renewcommand{\contentsname}{\vspace{-0.5cm} \textbf{Daftar Isi} \vspace{-2cm}}}

\begin{document}

% Cover Page
\begin{titlepage}
    \begin{center}
        \vspace*{0,2cm}

        \Huge
        \textbf{KB PEKAN 15-16 APLIKASI KOMPUTER}
        
        \vspace{1cm}
        
        \LARGE
        BAB LaTeX dan Markdown  
        
        \vspace{1cm}
        
        \includegraphics[width=0.5\textwidth]{Logo UNY.png}

        \vspace{1cm}
        
        \textbf{Nisrina Octaviani}\\
        22305144011\\
        Matematika B 2022
        
        \vspace{2cm}
        
        \Large
        \textbf{PRODI MATEMATIKA}\\
        \textbf{DEPARTEMEN PENDIDIKAN MATEMATIKA}\\
        \textbf{FAKULTAS MATEMATIKA DAN ILMU PENGETAHUAN ALAM}
        \textbf{UNIVERSITAS NEGERI YOGYAKARTA}\\
        \textbf{2023}
        
    \end{center}
\end{titlepage}


\newpage
\tableofcontents

\chapter{KB Pekan 2 : Pengenalan Software Euler Math Toolbox}
\input{KB Pekan 2_Nisrina Octaviani_22305144011}

\newpage
\chapter{KB Pekan 3-4 : Penggunaan Software EMT untuk Aljabar}
\input{KB Pekan 3&4_Nisrina Octaviani_22305144011}

\newpage
\chapter{KB Pekan 5-6: Penggunaan Software EMT untuk Plot 2D}
\input{KB Pekan 5&6_Nisrina Octaviani_22305144011}

\newpage
\chapter{KB Pekan 7-8: Penggunaan Software EMT untuk Plot 3D}
\input{KB Pekan 7&8_Nisrina Octaviani_22305144011}

\newpage
\chapter{KB Pekan 9-10: Menggunakan EMT untuk Kalkulus}
\input{KB Pekan 9&10_Nisrina Octaviani_22305144011}

\newpage
\chapter{KB Pekan 11-12: Menggunakan EMT untuk Geometri}
\input{KB Pekan 11&12_Nisrina Octaviani_22305144011}

\newpage
\chapter{KB Pekan 13-14: Menggunakan EMT untuk Statistika}
\input{KB Pekan 13&14_Nisrina Octaviani_22305144011}

